\section{Related Works}
\label{sec:related}
With the drastic development of home-use IoT systems, more and more studies were carried out to investigate the vulnerabilities of and the attacks on smart home systems; on the other hand, research on sensitive personal information leak by the smart home IoT systems, especially the smart health systems, just started to thrive. In this section, we provide a brief overview on the most related research.\\
%
\indent \textbf{Health-related IoT Security.}
The security of health-related devices and systems has gained more and more attention in recent years due to the rapid development and the tight personal health connection. Eberz \emph{et al.} \cite{eberz2017broken} presented a systematic attack against the ECG biometrics and demonstrated its effectiveness by applying it to a successful commercialized ECG biometric product, the Nymi Band; they transformed the ECG signals collected on different ECG devices by a mapping function such that the signal collected on one device can be mimicked by the signal collected by a different device so that they can use arbitrary ECG signals for spoofing attacks. Different from their work, we focused more on the vulnerabilities of the home-use NeuroSky EEG system framework and the leak of the sensitive information revealed by the raw EEG signals; we developed 5 PoC attacks through which victim's activities can be surmised by analyzing the captured raw EEG signals.\\
\indent Rahman \emph{et al.} \cite{rahman2013fit} investigated the security and privacy issues of Fitbit; after reversely engineering the ANT protocol for data communications, they generated various attacks including the data capture attack, injection attack, denial of service attack, and so on; they also proposed possible defense mechanisms against the proposed attacks. Li \emph{et al.} \cite{li2011hijacking} demonstrated several security attacks on a popular glucose monitoring and insulin delivery system available on the market; by recovering the radio protocol, they proposed various attack scenarios and performed two types of attacks, passive attack and active attack; then they analyzed the attack scenarios and generated two defense schemes. Compared to the two pieces of research work mentioned above, besides the security of the health-related device itself, we also focused on the sensitive information leakage revealed by the captured EEG signals.\\
%
\indent Martinovic \emph{et al.} \cite{martinovic2012feasibility} explored a research-use EEG devices, EPOC, as a potential attack intermediary to infer private information about their users; with the EEG signals captured from EPOC, they calibrated the classifiers on a set of training observations; their results supported the hypothesis that personal sensitive information such as bank account and password might be unintentionally leaked. In contrast to their research, we chose a much more popular setting with a home-use EEG IoT device, which has much fewer electrodes than the ones of EPOC; instead of using the EEG device as an attack intermediary, we directly explored the security vulnerabilities of the EEG device itself; based on the weaknesses, we implemented 5 PoC attacks to steal the EEG data from the device; with the reduced-featured EEG signals, we can infer the user's activity at approximately 70.55\% accuracy.\\
\indent
Alongside the security study on health-related IoT devices, many work focused on building smart health systems and platforms. In \cite{solanas2014smart} \cite{yang2014health}, the authors discussed the concept of health-related smart city and the home-based wellness platform. Mirza \emph{et al.} \cite{baig2013smart} provided an overview on the design and modeling of the current smart health monitoring systems. Coincidentally, ISLAM \emph{et al.} \cite{islam2015internet} investigated the e-Health technologies and reviewed the existing advanced network architectures for e-Health; they also analyzed the distinct security and privacy features of the e-health structures, proposed a collaborative security model, discussed the innovations in health care contexts, and addressed various e-health policies and regulations.\\
%
\indent\textbf{More General IoT Security.}
General IoT security research has been thriving in recent years. Fernandes \emph{et al.} \cite{fernandes2016security} studied the Samsung's SmartThings smart home system by performing a combination of static analysis, runtime testing, and manual analysis on a dataset of 499 SmartApps and 132 device handlers downloaded in source form; they discovered 2 design flaws that lead to over-privileges even with the privilege separation module implemented, and found that the SmartThings event subsystem does not sufficiently protect the events that carry sensitive information; by exploiting these design flaws, they developed 5 PoC attacks targetting the SmartThings smart home system.  Costin \emph{et al.} \cite{costin2014large} managed to unpack and analyze 26,275 embedded system firmware images crawled from the Internet; with static analysis, they discovered a total of 38 previously unknown vulnerabilities in over 693 firmware images. M$\ddot{u}$ller \emph{et al.} \cite{muller2017sok} conducted a large-scale analysis on the known printer attacks, and provided a general method for security analysis on printers; they also wrote a prototype software PRET to automatically evaluate 20 printer models and found that all of them are vulnerable to at least one of the known attacks. Manos \emph{et al.} \cite{antonakakis2017understanding} did a comprehensive analysis on Mirai's emergence and evolution. They analyzed how the botnet emerged, what classes of devices were affected, how Mirai variants evolved and competed for vulnerable hosts, and what types of attacks Mirai exploited.  Flavio \emph{et al.}  \cite{garcia2016lock} found that the entry systems of most VW Group vehicles maintain a few global master keys, which allows an adversary to clone a remote control and gain unauthorized access to a vehicle; they also presented a novel correlation-based attack on the Hitag2 rolling code scheme, which reveals the cryptographic key and thus enables an attacker to clone the remote control system. In contrast, our work focuses on the vulnerabilities of a specific type of home-use EEG system and its related sensitive information leakage. \\
%
\indent There also exists research targetting the communication protocols of the IoT systems. For example, Ronan \emph{et al.} \cite{ronen2017iot} discovered a bug in the Touchlink part of the ZigBee Light Link protocol implemented by Philips, and compromised the global AES-CCM key used to encrypt and authenticate a new firmware so that an attacker can replace the benign firmware with a malicious one over-the-air; with all these findings they implemented a smart light bulb worm which automatically spreads and controls all the Philips smart lights. Our work is along with a similar line except that we further analyzed the captured EEG signals rather than just taking full control of the devices.


