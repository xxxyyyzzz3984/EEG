\section{Defense Solutions}
\label{sec:defense}

In this section, we demonstrate defense solutions to mitigate our proposed attacks. For mitigating the remote attacks, we propose an OAuth-based framework to mitigate unauthorized applications exploting standard SDK and TGSP, and a signature-based solution to mitigate malicious SDK and malicious TGSP server. For mitigating the proximate attacks, we propose an encryption-based solution to eliminate possible eavesdropping and replay attacks.

\subsection{Solutions to Mitigate Remote Attacks}

\subsubsection{OAuth-based Framework}
The root cause of EEG data leakage to malicious applications through SDK and TGSP is because the ThinkGear framework lacks necessary authentication/authorization. Having noticed the cause, we realize that OAuth~\cite{hardt2012oauth}, a well-known authorization and authentication protocol, can be applied to the ThinkGear framework in order to eliminate attacks resulting from unauthorized malicious applications.

We demonstrate in details how OAuth 2.0 authorization code granting protocol can be applied to ThinkGear framework. To begin with, NeuroSky has to maintain a service-providing server on which all BCI-app developers have to register their BCI apps, and each BCI app is assigned with an app ID and an app secret (normally a random hash given by the service-providing server). If a BCI app wishes to access a user's EEG data, either through TGSP or SDK, the TGSP server or \texttt{thinkgear.dll}
