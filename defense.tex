\section{Defense Solutions}
\label{sec:defense}

In this section, we demonstrate defense solutions to mitigate our proposed attacks. For mitigating the remote attacks, we propose an OAuth-based framework to mitigate unauthorized applications exploting standard SDK and TGSP, and a signature-based solution to mitigate malicious SDK and malicious TGSP server. For mitigating the proximate attacks, we propose an encryption-based solution to eliminate possible eavesdropping and replay attacks.

\subsection{Solutions to Mitigate Remote Attacks}

\subsubsection{OAuth-based Framework}
The root cause of EEG data leakage to malicious applications through SDK and TGSP is because the ThinkGear framework lacks necessary authentication/authorization. Having noticed the cause, we realize that OAuth~\cite{hardt2012oauth}, a well-known authorization and authentication protocol, can be applied to the ThinkGear framework in order to eliminate attacks resulting from unauthorized malicious applications.

We demonstrate in details how OAuth 2.0 implicit granting protocol can be applied to ThinkGear framework. To begin with, NeuroSky has to maintain a service-providing server on which all BCI-app developers have to register their BCI apps, and each BCI app is assigned with an app ID and an app secret (normally a random hash given by the service-providing server). If a BCI app wishes to access a user's EEG data through TGSP or SDK, the TGSP server or \texttt{thinkgear.dll} then redirects the user to the service-providing server along with the requesting app ID and app secret. In this case, the service-providing server first checks if the requesting app ID and app secret match as the ones the developer registered. If they are not matched with any records, the request will be denied directly. If they are matched with one of the record, the user will be asked by the service-providing server whether or not the user wishes to grant the permission to the app to access his EEG data. If granted, the service-providing server passes an access token back to the TGSP server or the \texttt{thinkgear.dll}, and the TGSP server or the \texttt{thinkgear.dll} further passes the token to the BCI app. Then the BCI app can authenticate itself with the TGSP server or \texttt{thinkgear.dll} for EEG data using the access token. The overview of the protocol is shown as figure~\ref{fig:oauth}.

\begin{figure}[!htb]
\hspace*{2cm}
        \includegraphics[scale=0.4]{oauth.png}
	\caption{The OAuth 2.0 implicit granting protocol applied to ThinkGear framework. }
        \label{fig:oauth}
\end{figure}

\subsubsection{Signature-based Solution}
NeuroSky does not implement any tamper-resistent techniques, allowing an attacker to arbitrarily modify or fabricate its binary programs, i.e., \texttt{thinkgear.dll} and TGSP server. Digital signature is a widely-used mechanism for protecting applications from unauthorized modifications. Digital signature mechanism is consisted with two parts, generation and verification. In the generation step, NeuroSky first needs to generate a hash for each of the protected programs, i.e, \texttt{thinkgear.dll} and TGSP server. Then NeuroSky leverages one the asymmetric encryption algorithms by first generating a pair of public key and private key, then using the private key to encrypt the hashes, and finally inserting the encrypted hashes and the digital signature (the public key and the plaintext hashes), into the protected programs. In the verification step, before interacting with \texttt{thinkgear.dll} or the TGSP server, every BCI app should first extract the published digital signature and use the public key to decrypt the encrypted hashes, and compare whether the decrypted hashes are matched with the original plaintext hashes. If they are matched, it means the protected programs are intact, otherwise, it means that protected programs have been modified and the BCI apps should terminate the interactions. Digital signature is mature and widely-adopted in the industry, many off-the-shelf tools provide generation and verification services. SignTool is one of the most commonly-used tools developed by Microsoft for this purpose~\cite{signtool}.

\subsection{Solution to Mitigate Proximate Attack}
In order to inhibit the proximate attack, one could eliminate the possibility of replay attack. Simple encryption methods for radio freqency such as voice inversion, hopping inversion and rolling code inversion are easy to be cracked and are vulnerable to replay attacks. However, the computation power of the EEG devices limits the use of some complex encryption protocols, e.g., secure sockets layer (SSL). Having realized this, we suggest adopting advanced encryption standard (AES) for secure EEG data transmission. In order to do so, the NeuroSky headset first generates a random key as a secret key and establishes a Diffie-Hellman (DH) key exchange protocol at a certain channel to share the secret key. Then the headset can transmit the EEG data encrypted with the secret key using AES, then the RF dongle can decrypt the encrypted data with the secret key using AES to obtain the raw EEG data. By adopting the solution properly, NeuroSky can effectively inhibit the proximate attack.
