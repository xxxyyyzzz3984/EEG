\section{Defense Solutions and Side-channel Leakages}
\label{sec:defense}

In this section, we demonstrate defense solutions to mitigate our proposed attacks. For mitigating the remote attacks, we propose an OAuth-based framework to mitigate unauthorized applications exploting standard SDK and TGSP, and a signature-based solution to mitigate malicious SDK and malicious TGSP server. For mitigating the proximate attacks, we propose an encryption-based solution to eliminate possible eavesdropping and replay attacks. \textcolor{blue}{Note that since all these three defense methods are patching-based solutions which require the efforts from manufactures, therefore, we are not be able to implement them. We are hoping that the defense discussion of this section can serve as a security suggestion to most health-related IoT manufactures for the purpose of enhancing the security of their products. In addition, we also discuss the possible side-channel leakages even if the data are well protected after adopting our proposed defense solutions.}

\subsection{Solutions to Mitigate Remote Attacks}

\subsubsection{OAuth-based Framework}
The root cause of EEG data leakage to malicious applications through SDK and TGSP is because the EEG system framework lacks necessary authentication/authorization. Having noticed the cause, we realize that OAuth~\cite{hardt2012oauth}, a well-known authorization and authentication protocol, can be applied to the EEG system framework in order to eliminate attacks resulting from unauthorized malicious applications.

We demonstrate in details how OAuth 2.0 implicit granting protocol can be applied to EEG system framework.  \textcolor{blue}{The whole process of OAuth-patched EEG system framework is composed with 10 steps. (1) To begin with, NeuroSky has to maintain a service-providing server on which all BCI-app developers have to register their BCI apps, (2) and each BCI app is assigned with an app ID and an app secret (normally a random hash given by the service-providing server). (3) If a BCI app wishes to access a user's EEG data through TGSP or SDK, it would pass the app ID and app secret to the TGSP server or \texttt{thinkgear.dll}, (4) which then forward the app ID and app secret to the NeuroSky service-providing server. (5) In the mean time, user is redirected to the NeuroSky service-providing server along with the requesting app ID. (6) In this case, the service-providing server first checks if the requesting app ID and app secret match as the ones the developer registered. If they are not matched with any records, the request will be denied directly. If they are matched with one of the record, the user will be asked by the service-providing server whether or not the user wishes to grant the permission to the app to access his EEG data. (7) If granted, the service-providing server passes an access token back to the TGSP server or the \texttt{thinkgear.dll}, (8) and the TGSP server or the \texttt{thinkgear.dll} further forward the token to the BCI app. (9) Then the BCI app can authenticate itself with the TGSP server or \texttt{thinkgear.dll} using the access token, (10) in order to obtain EEG data. The overview of the protocol is shown as figure~\ref{fig:oauth}}.

\begin{figure}[!htb]
\hspace*{2cm}
        \includegraphics[scale=0.3]{oauth.png}
	\caption{The OAuth 2.0 implicit granting protocol applied to EEG system framework. }
        \label{fig:oauth}
\end{figure}

\subsubsection{Signature-based Solution}
NeuroSky does not implement any tamper-resistent techniques, allowing an attacker to arbitrarily modify or fabricate its binary programs, i.e., \texttt{thinkgear.dll} and TGSP server. Digital signature is a widely-used mechanism for protecting applications from unauthorized modifications. Digital signature mechanism is consisted with two parts, generation and verification. In the generation step, NeuroSky first needs to generate a hash for each of the protected programs, i.e, \texttt{thinkgear.dll} and TGSP server. Then NeuroSky leverages one the asymmetric encryption algorithms by first generating a pair of public key and private key, then using the private key to encrypt the hashes, and finally inserting the encrypted hashes and the digital signature (the public key and the plaintext hashes), into the protected programs. In the verification step, before interacting with \texttt{thinkgear.dll} or the TGSP server, every BCI app should first extract the published digital signature and use the public key to decrypt the encrypted hashes, and compare whether the decrypted hashes are matched with the original plaintext hashes. If they are matched, it means the protected programs are intact, otherwise, it means that protected programs have been modified and the BCI apps should terminate the interactions. Digital signature is mature and widely-adopted in the industry, many off-the-shelf tools provide generation and verification services. SignTool is one of the most commonly-used tools developed by Microsoft for this purpose~\cite{signtool}.

\subsection{Solution to Mitigate Proximate Attack}
In order to inhibit the proximate attack, one could eliminate the possibility of replay attack. Simple encryption methods for radio frequency such as voice inversion, hopping inversion and rolling code inversion are easy to be cracked and are vulnerable to replay attacks. However, the computation power of the EEG devices limits the use of some complex encryption protocols, e.g., secure sockets layer (SSL). Having realized this, we suggest adopting advanced encryption standard (AES) for secure EEG data transmission. \textcolor{blue}{In order to do so, the NeuroSky headset uses a secret key as a secret key to encrypt the EEG data using AES, then the RF dongle decrypt the encrypted data with the secret key using AES to obtain the raw EEG data. The key problem is the key distribution and management. Adopting  Diffie-Hellman key exchange protocol~\cite{diffie1976new}, agreeing on a pre-shared key, or maintaining a trusted sever and adopting attribute-based encryption~\cite{bethencourt2007ciphertext} are a new options. By adopting the solution properly, NeuroSky can effectively inhibit the proximate attack.}

{\color{blue}
\subsection{Possible Side-channel Leakages}
Even if the EEG data are properly protected under our proposed defense solutions, inference attacks exploiting side-channel information may be difficult to avoid. In this subsection, we discuss the possible side-channel-based inference attacks a remote attacker and a proximate attacker can exploit.

\subsubsection{Remote Side-channel Attacks}
Similar to our previous setting, a remote attacker has a malicious program running on the victim's PC. If adopting the OAuth protocol and the digital signature protection methods, the malicious program is unable to conduct any remote attacks mentioned in section~\ref{sec:attack}. However, the malicious program can capture any USB traffic coming in and out of RF dongle using USB protocol~\cite{usbprotocol}. The packet structures of USB protocol are somehow highly to the ones in network protocols, i.e, TCP and UDP. Li \emph{et al.}~\cite{li2016side} has demonstrated that a user's basic activities of daily living can be inferred from encrypted network video stream. Therefore, it is completely possible for an attacker to recover a user's brain status by monitoring the USB traffic coming in and out of the RF dongle.

\subsubsection{Proximate Side-channel Attacks}
Similar to our previous setting, a proximate attacker does not have access to the victim's PC, however, he is within certain range with the victim. By adopting encrypting the radio frequency, the attacker is not be able to conduct our record-and-replay attack to recover the victim's EEG data. However, the attacker can still record the encrypted radio waves. Radio waves are somehow highly similar to electromagnetic radiation. Song \emph{et al.}~\cite{song2016my} has demonstrated that an attacker can inferred what object a 3D printer prints by exploiting electromagnetic waves. Therefore, it is completely possible for an attacker to recover a user's brain status by monitoring the encrypted radio waves emitted from the headset.
}
