\section{Introduction}
\label{sec:introduction}
Health-related IoT devices have been constantly attracting public attentions in recent years. According to a most recent report, the market of healthcare IoT reaches 41.22 billion USD in 2017 and is projected to grow to 158.07 billion USD by the year of 2022 \cite{healthiotmarket}. Besides the dramatic growing trend of the gross benefits, health-related IoT market is also prospering in the aspect of diversity of its products. Devices like smart wristbands and smart scales are two of the most well-known products in this area with which a user can monitor her health conditions varying from heartbeats, sleep level, to muscle mass or even bone mass.\\
\indent On one hand, users gain tangible benefits from the health-related IoT devices. On the other hand, however, users are also exposed to new and unknown risks. In January 2017, Department of Homeland Security (DHS) confirmed that nearly 465,000 implanted heart pumping devices, which are actively used in hospitals all-round the US, are vulnerable to remote attacks \cite{heartpump}. For example, hackers can remotely hack into a patient's defibrillator and trigger irregular heart rhythms which can cause a cardiac failure \cite{heartpump}. In September 2017, DHS issued a warning stating that approximately 4,000 wireless syringe infusion pumping devices are vulnerable to remote attacks \cite{infusionhack}. For instance, attackers can exploit a loophole to murder a patient by remotely giving the patient overdose infusion \cite{infusionhack}. As one can see from these cases, different from compromising traditional IoT devices such as smart hues or thermostats that results in system failures, compromising health-related IoT devices can not only cause leakage of a user's health information, but also directly jeopardize the user's life.\\
\indent Unlike the professional medical IoT devices mentioned above whose security problems have been gradually explored and addressed by more and more researchers, the security of home-use health-related IoT devices is seriously under-investigated. Therefore, a natural question is raised: are emerging home-use health-related IoT devices, which have a large amount of users, also suffer from similar security threats? To answer this question, we performed, to the best of our knowledge, the \emph{first} security analysis on home-use electroencephalography (EEG) IoT devices. The reason we chose EEG devices for our research is because they have a drastic growing market which is projected to reach 1.40 billion USD by the year of 2025 \cite{2025eegdevicemarket}. Moreover, EEG data is one of the most important and sensitive human health data that can reveal an individual's sensitive health conditions. Therefore, it is urgent and critical to investigate the security vulnerabilities of home-use EEG devices.\\
\indent In this paper, we studied the security of home-use EEG devices targetting the ThinkGear AM (TGAM) module manufactured by NeuroSky. NeuroSky is the most well-known manufacturer of the home-use EEG devices since it is the first company to make brain wave devices for home use in the world \cite{firsthomeeeg}, and it still holds the largest market share of home-use EEG devices \cite{neuroskymarket}. TGAM is the exclusive brain wave sensor ASIC module developed by NeuroSky; it was elected as TIME Magazine's 100 Best Toys of All Time and has a potential to be widely adopted by home-use EEG IoT manufacturers \cite{tgammarket}. In this research, we demonstrated that based on the security flaws of the EEG system framework due to the coarse-grained implementation, an attacker is able to construct two sets of easy-to-exploit PoC attacks, which contain 4 remote attacks and one proximate attack, to actively steal a user's brain wave data. A remote attack does not require the attacker to be close to the victim but a carefully crafted malicious program is needed. The proximate attack, on the other hand, does not require any malicious program, but it has to be launched within a certain distance away from the victim. %Thus, an attacker can freely choose any of these two sets of attacks based on real world situation. 
Because the security flaw exploited by our proximate attack exists in a mandatory step for the EEG data retrieval, all the 156 brain-computer interface (BCI) apps appearing in the NeuroSky App store, the official App store for TGAM devices, are vulnerable to this attack. We also conducted an empirical static binary analysis against all the 31 free apps in the NeuroSky App store and found that all of them are vulnerable to at least one remote attacks. %\\, with 32\% of them are vulnerable to both attacks. \\
\indent In addition, we studied the potential privacy leakage problem from the EEG data collected by the TGAM devices. It is well researched that rich-featured EEG signals collected by strict medical-use devices or research-use devices can reveal a user's sensitive information, e.g., focal disorders \cite{eegdiagnosis} \cite{dauwels2010diagnosis} and sleep levels \cite{nakamura2017automatic}. However, whether or not reduced-featured EEG data collected by home-use EEG devices can also pose similar privacy threats remains unexplored. In this paper, we proposed a novel recurrent convolutional neural network model (RCNN) targeting the reduced-featured EEG data collected by home-use EEG IoT devices to infer a user's focusing activities; our evaluation results over the realworld EEG data revealed that the proposed RCNN has an accuracy as high as 70.55\%, which significantly outperforms the other 11 most widely-used learning models.\\
%\indent Note that Martinovic \emph{et al.} \cite{martinovic2012feasibility} studied the possible side-channel leak from brain-computer interface (BCI) apps. Their work has the following two key differences from ours. First, we studied the security of home-use EEG IoT devices on both device-level and app-level, while they mainly studied the one on app-level. Second, the types of information inferred by their model are different the ones inferred by ours. Their model inferred a user's static information, e.g., the bank of a user's credit card, while our model inferred a user's active private information, e.g., the activity a user is currently focusing on. Third, Martinovic's \emph{et al.} leveraged EPOC device which is a research-use device and is not publicly available on the market for normal consumers as we will mention in section~\ref{sec:background}, even though the EPOC device collects much more features than home-use EEG IoT devices. While in our research, we took a step further to explore inference attacks on reduced-featured EEG data collected by home-use EEG IoT devices.\\
%\indent Note that Martinovic \emph{et al.} \cite{martinovic2012feasibility} studied the possible side-channel leak from brain-computer interface (BCI) apps. Their work has the following two key differences from ours. First, they focused on whether sensitive information can be deduced from the EEG data while we focused on how vulnerable the framework of home-use EEG IoT devices is. Second, Martinovic's \emph{et al.} leveraged the EPOC device which is a research-use device that is not publicly available in the market for normal home-use as we mention in section~\ref{sec:background}; the EPOC device collects much more features than home-use EEG IoT devices. While in our research, we took a step further to explore inference attacks on reduced-featured EEG data collected by home-use EEG IoT devices.\\
\indent \textbf{Our Contributions.} In summary, we have the following contributions in this paper:
\begin{itemize}
\item To the best of our knowledge, we proposed the first security analysis against home-use EEG IoT devices. We demystified the NeuroSky EEG system framework, which is not documented by the official NeuroSky documentations.

\item We identified the security flaws in the NeuroSky EEG system framework, based on which we constructed and fully implemented two sets of easy-to-exploit attacks to steal a user's brain wave data. %An attacker can freely choose any of these two sets of attacks according to real world situations. 
We also analyzed the BCI apps on the NeuroSky App store and discovered that all apps are vulnerable to the proximate attack, and that all the free apps are vulnerable to at least one remote attack. % one or both of the remote attacks: the malicious TGSP server attack and the malicious SDK attack.

\item We finally proposed a novel deep learning model to infer a user's activities based on the stolen reduced-featured EEG data with an inference accuracy as high as 70.55\%, which significantly outperforms other well-known learning models such as SVM, decision tree, and adaboost.
\end{itemize}

\indent \textbf{Paper Organization.} The rest of the paper is organized as follows. Section \ref{sec:related} outlines the most related work. Section~\ref{sec:background} introduces the preliminary knowledge on EEG, the types of different EEG devices, and our threat model. Section \ref{sec:framework} details the demystified EEG system framework. Section \ref{sec:attack} demonstrates the security flaws of the framework and the implementations of our attacks based on these flaws. Section \ref{sec:inference} presents our deep learning model for the reduced-featured EEG inference attack. Section \ref{sec:performance} reports the performance of our attacks and Section   \ref{sec:conclusion} concludes the paper.

%Finally, we wish that our research can serve as a security lesson to related parties and may offer an  to strengthen the security design for health-related IoT devices.